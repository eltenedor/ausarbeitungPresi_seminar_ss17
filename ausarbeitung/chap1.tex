\section{Kategorientheoretische Grundlagen (Fortsetzung)}

In diesem Abschnitt füllen wir das Vokabelheft mathematischer Definitionen mit weiteren Begriffen aus der Kategorientheorie.

\subsection{Funktoren, universelle Morphismen und natürliche Transformationen}

Wie schon in der Einleitung angekündigt, werden Funktoren in diesem Teil des Seminars eine prominente Rolle spielen, denn wir wollen unterschiedliche Kategorien zueinander in Relation setzen und Abbildungen die dies auf \emph{natürliche} Weise schaffen, nennen wir Funktoren.

\begin{defn}[Covarianter Funktor]
  Seien $\C$ und $\De$ Kategorien und $\F_1 \colon |\C| \to |\De|$ and $\F_2 \colon \Mor_\C\to \Mor_\De$. 
  Dann nennen wir das Quadrupel $\F = (\C, \De, \F_1, \F_2)$ einen (\emph{covarianten}) \emph{Funktor} von $\C$ nach $\De$, falls folgende Bedingungen erfüllt sind
  \begin{enumerate}[F1)]
    \item $f \in [A, B]_\C$ impliziert $\F(f) \in [\F(A), \F(B)]_\De$.
    \item $\F(f \circ g) = \F(f) \circ \F(g)$, falls $f \circ g$ definiert ist.
    \item $\F(1_A) = 1_{\F(A)}$ für alle $A \in |\C|$.
  \end{enumerate}
  Abkürzend schreiben wir im Folgenden auch $\F \colon \C \to \De$.
\end{defn}

Aufgrung der Eigenschaften F2) und F3) bezeichnet man Funktoren machmal auch als Homomorphismen von Morphismen.

Ein \emph{contravarianter Funktor} $\F \colon \C \to \De$ ist gerate ein covarianter Funktor von $\C^* \to \De$, was gerade bedeutet, dass die zusätzlich zu F1) folgende modifizierten Bedingungen gelten:
\begin{enumerate}[F1')]
  \item $f \in [A, B]_\C$ impliziert $\F(f) \in [\F(B), \F(A)]_\De$.
  \item $\F(f \circ g) = \F(g) \circ \F(f)$, falls $f \circ g$ existiert.
\end{enumerate}

Wir beleben den Begriff des Funktors nun, indem wir bekannte Sachverhalte kategorientheoretisch unter die Lupe nehmen.

\begin{ex}
\begin{enumerate}[a)]
    \item Konstanter Funktor: Sind $\C$ und $\De$ Kategorien und $X \in |\De|$.
      So lässt sich für alle $A \in |\C|$ und alle $f \in \Mor_\C$ durch $\F(A) = X$ und $\F(f) = 1_X$ ein Funktor definieren, welcher sowohl covariant als auch contravariant ist.
    \item Vergissfunktor: Ist $\C$ ein (topologisches) Konstrukt, so lässt sich ein Funktor $\F \to \Set$ definieren durch $\F((X, \xi)) = X$ und $\F(f) = f$.
    \item Dualisierender Funktor $\Delta_\C$: Es lässt sich ein Funktor $\F \colon \C \to \C^*$ definieren durch $\F(X) = X$ und $F(f) = f^*$. Dieser ist natürlich contravariant.
    \item Dualer Funktor: Ist $\F \colon \C \to \De$ ein Funktor, so erhält man den zugehörigen dualen Funtor als $\Delta_{\De} \circ F \circ \Delta_{\C^*}$
    \item Inklusionsfunktor: Sei $\C$ eine Kategorie und $\A$ eine \emph{Unterkategorie}, also eine Kategorie für die gilt
      \begin{enumerate}
        \item $|\A| \subset |\C|$,
        \item $[A, B]_\A \subset [A,B]_\C$ für alle $(A, B) \in |\A| \times |\A|$,
        \item Die Komposition von Morphismen in $\A$ stimmt mit der Komposition in $\C$ überein und die der Identitätsmorphismus ist auch derselbe.
      \end{enumerate}
      Git zu alledem sogar $[A, B]_\A = [A, B]_\C$, so bezeichnen wir $\A$ als \emph{volle} Unterkategorie.
    \item Identitätsfunktor $\I_\C$: Dieser Funktor $\F \colon \C \to \C$ wird durch $\F(X) = X$ und $\F(f) = f$ definiert.
  \end{enumerate}
\end{ex}

\begin{defn}[Universelle Abbildung]
  Seien $\A$ und $\B$ Kategorien, $\F \colon \A \to \B$ ein Funktor und $B \in |\B|$.
  Ein Paar $(u, A)$ mit $A \in |\A|$ und $u \colon B \to \F(A)$ heißt \emph{universelle Abbildung für $B$ bezüglich $\F$}, falls für alle $A' \in |\A|$ und alle $f \colon B \to \F(A')$ genau ein $\A$-Morphismus $\overline f \colon A \to A'$ existiert so dass das Diagramm
  $$
  \begin{tikzcd}[row sep=2.5em]
    B \arrow[rd,"u"] \arrow[rr, "f"] &&\F(A') \\
    &\F(A) \arrow[ru, "\F(\overline{f})"]
  \end{tikzcd}
  $$
  kommutiert.
  Entsprechend bezeichnet man ein Paar $(A,u)$ mit $A \in |\A|$ und $u \colon \F(A) \to B$ als \emph{co-universelle Abbildung für $B$ bezüglich $\F$}, falls $(u^*, A)$ eine universelle Abbildung für $B$ bezüglich des Funktors $\F^* \colon \A^* \to \B^*$ ist.
  Dies bedeutet, dass für alls $A' \in |\A|$ und jeden $\B$-Morphismus $f \colon \F^(A') \to B$ ein eindeutiger $\A$-Morphismus existiert, so dass das Diagramm
  $$
  \begin{tikzcd}[row sep=2.5em]
    B  &&  \arrow[ll, "f"] \F(A')\arrow[ld, "\F(\overline{f})"]  \\
    & \arrow[lu,"u"]\F(A) & 
  \end{tikzcd}
  $$
  kommutiert.
\end{defn}

Im folgenden Lemma beschreiben wir das Verhalten (co-)universeller Abbildungen unter Verknüpfung mit Isomorphismen.

\begin{lem}
  \label{lem:universalCircIso}
  Seien $\A$ und $\B$ Kategorien, $\F \colon \A \to \B$ ein Funktor und $B \in |\B|$ und $(u,A)$ eine universelle Abbildung für $B$ bezüglich $\F$.
  Sei nun $v \colon A \to \underline A$ ein $\A$-Isomorphismus, dann ist auch $(\F(v) \circ u, \underline A )$ eine universelle Abbildung für $B$ bezüglich $\F$.

  Ist $(A, u)$ eine couniverselle Abbildung für $B$ bezüglich $\F$, so ist auch $(\underline A, u \circ \F(v^{-1}))$ eine couniverselle Abbildung für $B$ bezüglich $\F$.
\end{lem}

\begin{proof}
  Sei $f \colon B \to \F(A')$ ein $\B$-Morphismus. 
  So existiert aufgrund der Eigenschaften von $u$ genau ein $\A$-Morphismus $\overline f \colon A \to A'$ mit $f = \F(\overline f) \circ u$.
  Aufgrund der Eindeutigkeit von $f$ existiert somit genau ein $g \coloneqq v^{-1} \circ \overline f \colon \underline A \to A'$, sodass das Diagramm   
  $$
  \begin{tikzcd}[row sep=2.5em]
    B \arrow[rd,"u"] \arrow[rr, "f"] &&\F(A') \\
    &\F(A) \arrow[ru, "\F(\overline{f})"] \arrow[rd,"\F(v)"] \\
    &&\F(\underline A) \arrow[uu,"\F(g)"]
  \end{tikzcd}
  $$
  mitsamt seiner Unterdiagramme kommutiert.
  Über ein analoges Argument zeigt man, dass im Falle einer couniversellen Abbildung das Diagramm
  $$
  \begin{tikzcd}[row sep=2.5em]
    B  & & \F(A') \arrow[ll, "f"]  \\
    & \F(A)\arrow[ru, "\F(\overline{f})"] \arrow[lu,"u"] \arrow[rd, "\F(v)"]  & \\
    & & \F(\underline A) \arrow[uu, "\F(g)"]
  \end{tikzcd}
  $$
  kommutiert.
\end{proof}

wir zeigen nun gewissermaßen die Umkehrung des vorangehenden Lemmas, nämlich, dass universelle Abbildungen bereits eindeutig bis auf Isomorphie sind.

\begin{prop}[\cite{preuss}, 2.1.6]
  Seien $(u,A)$ und $(u',A')$ universelle Abbildungen für $B \in |\B|$ bezüglich $\F \colon \A \to \B$.
  Dann existiert ein Isomorphismus $f \colon A \to \A'$, sodass das Diagramm
  $$
  \begin{tikzcd}[row sep=2.5em]
    B \arrow[rd,"u'"] \arrow[rr, "u"] &&\F(A) \\
    &\F(A') \arrow[ru, "\F(\overline{f})"]
  \end{tikzcd}
  $$
  kommutiert.
\end{prop}

Widmen wir unsere Aufmerksamkeit nun ein paar famosen Beispielen, um die eigeführten Prinzipien bei der Arbeit zu bestaunen.

\begin{ex}
  \begin{enumerate}
    \item T0-ifizierung: 
    \item Stone-Cech Kompaktifizierung: Sei $\Tych$ die Kategorie der Tychonoff-Räume, und $\CompHaus$ die Kategorie der kompakten Hausdorff-Räume.
      Ferner bezeichnen wir für $X \in |\Tych|$ mittels $\beta(X)$ seine Stone-\v{C}ech-Kompatifizierung und mit $e_X \colon X \to \beta(X)$ die entsprechende Einbettung. Dann ist nach dem Satz von Stone-\v{C}ech \cite[5.4.8]{bartsch} das Paar $(e_X, \beta(X))$ eine universelle Abbildung bezüglich des Inklusionsfunktors $\F_e \colon \CompHaus \to \Tych$.
      Ein entsprechend angepasstes kommutatives Diagramm liefert Gewissheit. Hierbei sei $Y \in \CompHaus$ und $f \in [X, \F_e(Y)]_{\Tych}$:
      $$
      \begin{tikzcd}[row sep=2.5em]
        X \arrow[rd,"e_X"] \arrow[rr, "f"] &&\F_e(Y) = Y \\
        &\F_e(\beta(X)) = \beta(X) \arrow[ru, "\F_e(\overline{f})"]
      \end{tikzcd}
      $$
    \item Vergissfunktor
  \end{enumerate}
\end{ex}

\begin{defn}[Natürliche Transformationen/Morphismen von Funktoren]
  
\end{defn}

\subsection{Adjungierte Funktoren}

Morphismen zwischen Identitätsfunktor und einer Verkettung von Funktoren

\begin{thm}[\cite{preuss}, 2.1.12]
  Sei $\F \colon \A \to B$ ein Funktor mit der Eigenschaft, dass für alle $B \in |\B|$ eine universelle Abbildung $(u_B, A_B)$ bezüglich $F$ existiert.
  Dann existiert genau ein Funktor $\G \colon B \to \A$, sodass Folgendes gilt:
  \begin{enumerate}[(1)]
    \item $\G(B) = A_B$ für alle $B \in |\B|$.
    \item $u = (u_B) \colon \I_B \to \F \circ \G$ ist eine natürliche Transformation.
  \end{enumerate}
\end{thm}

\begin{kor}[\cite{preuss}, 2.1.12]
  Es existiert genau eine natürliche Transformationa $v = (v_A) \colon \G \circ F \to \I_A$, sodass das Folgende gilt:
  \begin{enumerate}[(a)]
    \item $\F(v_A) \circ u_{\F(A)} = \1_{\F(A)}$ für alle $A \in |\A|$.
    \item $v_{\G(B)} \circ \G(u_B) = \1_{\G(B)}$ für alle $B \in |\B|$.
  \end{enumerate}
\end{kor}

\begin{defn}[Linksadjungierter Funktor]
  Sind $\F \colon \A \to \B$ und $\G \colon \B \to \A$ Funktoren und $u = (u_B) \colon \I_\B \to \F \colon \G$ sowie $v = (v_A) \colon \G \circ \F \to \I_\A$ natürliche Transformationen mit den Eigenschaften
  \begin{enumerate}[(1)]
    \item $\F(v_A) \circ u_{\F(A)} = \1_{\F(A)}$ für alle $A \in |\A|$ und
    \item $v_{\G(B)} \circ \G(u_B) = \1_{\G(B)}$ für alle $B \in |\B|$,
  \end{enumerate}
  so nennen wir $\G$ den \emph{zu $\F$ linksadjungierten Funktor} und analog nennen wir $\F$ den \emph{zu $\G$ rechtsadjungierten Funktor}.
  Das Paar $(\G, \F)$ nennen wir ein \emph{Paar adjungierter Funktoren}.
\end{defn}

\begin{thm}[\cite{preuss}, 2.1.15]
  Ist $\G \colon \B \to \A$ ein zu $\F \colon \A \to \B$ linksadjungierter Funktor und $u = (u_B) \colon \I_\B \to \F \circ \G$ eine zugehörige natürliche Transformation, dann ist für alle $B \in |B|$ das Paar $(u_B, \G(B))$ eine universelle Abbildung bezüglich $\F$.
\end{thm}

\begin{bem}[Adjungierte Situation]
    
\end{bem}

\begin{ex}
  \begin{itemize}
    \item T0-ifizierung
    \item Stone-Cech
    \item Vergissfunktor
  \end{itemize}
\end{ex}


