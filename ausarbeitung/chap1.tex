\section{Kategorientheoretische Grundlagen (Fortsetzung)}

In diesem Abschnitt füllen wir das Vokabelheft mathematischer Definitionen mit weiteren Begriffen aus der Kategorientheorie.

\subsection{Funktoren, universelle Morphismen und natürliche Transformationen}

Wie schon in der Einleitung angekündigt, werden Funktoren in diesem Teil des Seminars eine prominente Rolle spielen, denn wir wollen unterschiedliche Kategorien zueinander in Relation setzen und Abbildungen, die dies auf \emph{natürliche} Weise schaffen, nennen wir Funktoren.

\begin{defn}[Covarianter Funktor]
  Seien $\C$ und $\De$ Kategorien und $\F_1 \colon |\C| \to |\De|$ und $\F_2 \colon \Mor_\C\to \Mor_\De$. 
  Dann nennen wir das Quadrupel $\F = (\C, \De, \F_1, \F_2)$ einen (\emph{covarianten}) \emph{Funktor} von $\C$ nach $\De$, falls folgende Bedingungen erfüllt sind:
  \begin{enumerate}[F1)]
    \item $f \in [A, B]_\C$ impliziert $\F(f) \in [\F(A), \F(B)]_\De$.
    \item $\F(f \circ g) = \F(f) \circ \F(g)$, falls $f \circ g$ definiert ist.
    \item $\F(1_A) = 1_{\F(A)}$ für alle $A \in |\C|$.
  \end{enumerate}
  Abkürzend schreiben wir im Folgenden auch $\F \colon \C \to \De$.
\end{defn}

Aufgrund der Eigenschaften F2) und F3) bezeichnet man Funktoren manchmal auch als Homomorphismen von Morphismen.

Ein \emph{contravarianter Funktor} $\F \colon \C \to \De$ ist gerate ein covarianter Funktor von $\C^* \to \De$, was gerade bedeutet, dass folgende modifizierte Bedingungen gelten:
\begin{enumerate}[F1')]
  \item $f \in [A, B]_\C$ impliziert $\F(f) \in [\F(B), \F(A)]_\De$.
  \item $\F(f \circ g) = \F(g) \circ \F(f)$, falls $f \circ g$ existiert.
\end{enumerate}

Wir beleben den Begriff des Funktors nun, indem wir bekannte Sachverhalte kategorientheoretisch unter die Lupe nehmen.

\begin{ex}
  \label{ex:Example}
\begin{enumerate}[a)]
    \item Konstanter Funktor: Sind $\C$ und $\De$ Kategorien und $X \in |\De|$, so lässt sich für alle $A \in |\C|$ und alle $f \in \Mor_\C$ durch $\F(A) = X$ und $\F(f) = 1_X$ ein Funktor definieren, welcher sowohl covariant als auch contravariant ist.
    \item Vergissfunktor: Ist $\C$ ein (topologisches) Konstrukt, so lässt sich ein Funktor $\F \to \Set$ definieren durch $\F((X, \xi)) = X$ und $\F(f) = f$.
    \item Dualisierender Funktor $\Delta_\C$: Es lässt sich ein Funktor $\F \colon \C \to \C^*$ definieren durch $\F(X) = X$ und $F(f) = f^*$. Dieser ist natürlich contravariant.
    \item Dualer Funktor: Ist $\F \colon \C \to \De$ ein Funktor, so erhält man den zugehörigen dualen Funtor als $\Delta_{\De} \circ F \circ \Delta_{\C^*}$
    \item Inklusionsfunktor: Sei $\C$ eine Kategorie und $\A$ eine \emph{Unterkategorie}, also eine Kategorie für die gilt
      \begin{enumerate}
        \item $|\A| \subset |\C|$,
        \item $[A, B]_\A \subset [A,B]_\C$ für alle $(A, B) \in |\A| \times |\A|$,
        \item Die Komposition von Morphismen in $\A$ stimmt mit der Komposition in $\C$ überein und die der Identitätsmorphismus ist auch derselbe.
      \end{enumerate}
      Gilt zu alledem sogar $[A, B]_\A = [A, B]_\C$, so bezeichnen wir $\A$ als \emph{volle} Unterkategorie.
    \item Identitätsfunktor $\I_\C$: Dieser Funktor $\F \colon \C \to \C$ wird durch $\F(X) = X$ und $\F(f) = f$ definiert.
  \end{enumerate}
\end{ex}

Wir sehen also: Unsere Welt ist voller Funktoren! Um den Kreis der Verdächtigen, die wir später noch genauer unter die Lupe nehmen wollen, etwas einzuschränken, führen wir einen speziellen Typ Morphismus ein, der es uns im weiteren Verlauf gestatten wird, verdächtige kategorientheoretische Sachverhalte als funktorielle Zusammenhänge zu entlarven.

\begin{defn}[Universelle Abbildung]
  Seien $\A$ und $\B$ Kategorien, $\F \colon \A \to \B$ ein Funktor und $B \in |\B|$.
  Ein Paar $(u, A)$ mit $A \in |\A|$ und $u \colon B \to \F(A)$ heißt \emph{universelle Abbildung für $B$ bezüglich $\F$}, falls für alle $A' \in |\A|$ und alle $f \colon B \to \F(A')$ genau ein $\A$-Morphismus $\overline f \colon A \to A'$ existiert, sodass das Diagramm
  $$
  \begin{tikzcd}[row sep=2.5em]
    B \arrow[rd,"u"] \arrow[rr, "f"] &&\F(A') \\
    &\F(A) \arrow[ru, "\F(\overline{f})"]
  \end{tikzcd}
  $$
  kommutiert.
  Entsprechend bezeichnet man ein Paar $(A,u)$ mit $A \in |\A|$ und $u \colon \F(A) \to B$ als \emph{co-universelle Abbildung für $B$ bezüglich $\F$}, falls $(u^*, A)$ eine universelle Abbildung für $B$ bezüglich des dualen Funktors $\F^* \colon \A^* \to \B^*$ ist.
  Dies bedeutet, dass für alle $A' \in |\A|$ und jeden $\B$-Morphismus $f \colon \F(A') \to B$ ein eindeutiger $\A$-Morphismus $\overline f \colon A' \to A$ existiert, sodass das Diagramm
  $$
  \begin{tikzcd}[row sep=2.5em]
    B  &&  \arrow[ll, "f"] \F(A')\arrow[ld, "\F(\overline{f})"]  \\
    & \arrow[lu,"u"]\F(A) & 
  \end{tikzcd}
  $$
  kommutiert.
\end{defn}

Man beachte, dass der Morphismus $\F(\overline f)$ keineswegs eindeutig festgelegt sein muss. Dies ist im Allgemeinen erst dann der Fall, wenn $\F$ ein \emph{treuer} Funktor ist.

Im folgenden Lemma beschreiben wir das Verhalten (co-)universeller Abbildungen unter Verknüpfung mit Isomorphismen.

\begin{lem}
  \label{lem:universalCircIso}
  Seien $\A$ und $\B$ Kategorien, $\F \colon \A \to \B$ ein Funktor und $B \in |\B|$ und $(u,A)$ eine universelle Abbildung für $B$ bezüglich $\F$.
  Sei nun $v \colon A \to \underline A$ ein $\A$-Isomorphismus, dann ist auch $(\F(v) \circ u, \underline A )$ eine universelle Abbildung für $B$ bezüglich $\F$.

  Ist $(A, u)$ eine couniverselle Abbildung für $B$ bezüglich $\F$, so ist auch $(\underline A, u \circ \F(v^{-1}))$ eine couniverselle Abbildung für $B$ bezüglich $\F$.
\end{lem}

\begin{proof}
  Sei $f \colon B \to \F(A')$ ein $\B$-Morphismus. 
  So existiert aufgrund der Eigenschaften von $u$ genau ein $\A$-Morphismus $\overline f \colon A \to A'$ mit $f = \F(\overline f) \circ u$.
  Aufgrund der Eindeutigkeit von $\overline f$ existiert somit auch genau ein $g \coloneqq v^{-1} \circ \overline f \colon \underline A \to A'$, sodass das Diagramm   
  $$
  \begin{tikzcd}[row sep=2.5em]
    B \arrow[rd,"u"] \arrow[rr, "f"] &&\F(A') \\
    &\F(A) \arrow[ru, "\F(\overline{f})"'] \arrow[rd,"\F(v)"] \\
    &&\F(\underline A) \arrow[uu,"\F(g)"']
  \end{tikzcd}
  $$
  mitsamt seiner Unterdiagramme kommutiert.
  Über ein analoges Argument zeigt man, dass im Falle einer couniversellen Abbildung $(A,u)$ das Diagramm
  $$
  \begin{tikzcd}[row sep=2.5em]
    B  & & \F(A') \arrow[ll, "f"]\arrow[ld, "\F(\overline{f})"] \arrow[dd, "\F(g)"] \\
    & \F(A) \arrow[lu,"u"] \arrow[rd, "\F(v)"]  & \\
    & & \F(\underline A) 
  \end{tikzcd}
  $$
  kommutiert. Hierbei ist $g \coloneqq v \circ \overline f$.
\end{proof}

Wir halten nun gewissermaßen die Umkehrung des vorangehenden Lemmas fest, nämlich, dass universelle Abbildungen bereits eindeutig bis auf Isomorphie sind.

\begin{prop}[\cite{preuss}, 2.1.6]
  Seien $(u,A)$ und $(u',A')$ universelle Abbildungen für $B \in |\B|$ bezüglich $\F \colon \A \to \B$.
  Dann existiert ein Isomorphismus $f \colon A \to \A'$, sodass das Diagramm
  $$
  \begin{tikzcd}[row sep=2.5em]
    B \arrow[rd,"u'"] \arrow[rr, "u"] &&\F(A) \\
    &\F(A') \arrow[ru, "\F(\overline{f})"]
  \end{tikzcd}
  $$
  kommutiert.
\end{prop}

Widmen wir unsere Aufmerksamkeit nun einem famosen Beispiel, um die eigeführten Prinzipien bei der Arbeit zu bestaunen.

\begin{ex}[Stone-\v{C}ech Kompaktifizierung]
       Sei $\Tych$ die Kategorie der Tychonoff-Räume, und $\CompHaus$ die Kategorie der kompakten Hausdorff-Räume.
      Ferner bezeichnen wir für $X \in |\Tych|$ mittels $\beta(X)$ seine Stone-\v{C}ech-Kompatifizierung und mit $e_X \colon X \to \beta(X)$ die entsprechende Einbettung. Dann ist nach dem Satz von Stone-\v{C}ech \cite[5.4.8]{bartsch} das Paar $(e_X, \beta(X))$ eine universelle Abbildung bezüglich des Inklusionsfunktors $\F_e \colon \CompHaus \to \Tych$.
      Ein entsprechend angepasstes kommutatives Diagramm liefert Gewissheit:
      $$
      \begin{tikzcd}[row sep=2.5em]
        X \arrow[rd,"e_X"] \arrow[rr, "f"] &&\F_e(Y) = Y \\
        &\F_e(\beta(X)) = \beta(X) \arrow[ru, "\F_e(\overline{f})"]
      \end{tikzcd}
      $$
      Hierbei sei $Y \in \CompHaus$ und $f \in [X, \F_e(Y)]_{\Tych}$.
\end{ex}

Funktoren für sich alleine sind ja schon interessante Kreaturen, aber wie lassen sich Beziehungen unterschiedlicher Funktoren zueinander beschreiben?
Stellt man sich die Frage, ob man nicht auch Funktoren als Objekte eine Kategorie auffassen kann, so kommt sofort die Frage auf, welches die richtigen Morphismen zwischen Funktoren sind. 

\begin{defn}[Natürliche Transformationen]
  Seien $\C$ und $\De$ Kategorien und $\F, \G \colon \C \to \De$ Funktoren.
  \begin{enumerate}[1)]
    \item Eine Familie $\eta = (\eta_A)_{A \in |\C|}$ mit $\eta_A \in [\F(A), \G(A)]_\De$ für alle $\A \in |\C|$ heißt \emph{natürliche Transformation}, falls für alle $(A,B) \in |\C| \times |\C|$ und alle $f \in [A,B]_\C$ das rechte Diagramm
      $$
      \begin{tikzcd}[row sep=2.5em]
        A\arrow[d,"f"] & & \F(A)\arrow[r,"\eta_A"]\arrow[d,"\F(f)"'] & \G(A)\arrow[d,"\G(f)"] \\
        B & & \F(B)\arrow[r,"\eta_B"] & \G(B)
      \end{tikzcd}
      $$
      kommutiert. Ist $\eta$ eine natürliche Transformation, schreiben wir im Folgenden auch kurz $\eta \colon \F \to \G$.
    \item Eine natürliche Transformation $\eta \colon \F \to \G$ heißt \emph{natürliche Äquivalenz}, falls für alle $A \in |\C|$ der Morphismus $\eta_A$ ein Isomorphismus ist.
    \item $\F$ und $\G$ heißen \emph{natürlich äquivalent}, oder kurz $\F \approx \G$, wenn eine natürliche Äquivalenz $\eta \colon \F \to \G$ existiert.
  \end{enumerate}
\end{defn}

Weiteres zur Kategorie der Funktoren lässt sich in \cite{brandenburg2016einfhrung} nachlesen.

\subsection{Adjungierte Funktoren}

Ein Funktor $\F \colon \C \to \De$ ist natürlich nicht immer ein Isomorphismus, d.h. es existiert nicht unbedingt ein Funktor $\G \colon \De \to \De$ mit
$$
\F \circ \G = \I_\De \quad\text{und}\quad \G \circ \F = \I_\C.
$$

Sind Funktoren nicht invertierbar, können sie zumindest \emph{fast} invertierbar mittels einer Pseudoinversen $\G \colon \De \to \C$ sein.
Dies ist ein Funktor mit 
$$\F \circ \G \approx \I_\De \quad\text{und}\quad \G \circ \F \approx \I_\C.$$

Auch das ist nicht immer der Fall und folglich ist man bestrebt zu untersuchen, ob ein Funktor nicht zumindest \emph{fast fast} invertierbar ist, im Sinne dass ein weiterer Funktor $\G \colon \De \to \C$ existiert, sodass zumindest natürliche Transformationen
$$
\eta \colon \F \circ \G \to \I_\De \quad\text{und}\quad \xi \colon \G \circ \F \to \I_\C
$$
zur Stelle sind.
Um ebendiese Funktoren dreht es sich in diesem Unterabschnitt.

Zunächst wollen wir jedoch die im vorangehenden Kapitel neu eingeführten Begriffe \emph{natürliche Transformation} und \emph{universelle Abbildung} mit einander verheiraten.

\begin{thm}[\cite{preuss}, 2.1.12]
  Sei $\F \colon \A \to B$ ein Funktor mit der Eigenschaft, dass für alle $B \in |\B|$ eine universelle Abbildung $(u_B, A_B)$ bezüglich $F$ existiert.
  Dann existiert genau ein Funktor $\G \colon B \to \A$, sodass Folgendes gilt:
  \begin{enumerate}[(1)]
    \item $\G(B) = A_B$ für alle $B \in |\B|$.
    \item $u = (u_B) \colon \I_B \to \F \circ \G$ ist eine natürliche Transformation.
  \end{enumerate}
\end{thm}

Moralisch gesehen sollte der Morphismus-Part des entstehenden Funktors gegeben sein durch $\G(f) = \overline f$ und dies wird für die im Späteren betrachteten topologischen Konstrukte sogar nicht nur intuitiv richtig sein. 
Die saubere Realität sieht anders aus, wie der folgende Beweis zeigen soll.

\begin{proof}
  Die Aussage des Theorems beinhaltet bereits den Objekt-Teil des Funktors $\G$, den wir auch brav so definieren, d.h. für alle $B \in |B|$ setzen wir $\G(B) = A_B$.

  Betrachten wir nun das Diagramm
  $$
    \begin{tikzcd}[row sep=2.5em]
      \I_\B(B)  = B \arrow[r,"u_B"] \arrow[d,"f"'] & \F(A_B)     = \F(\G(B)) \arrow[d,"\F(\widetilde f)"] 
      \\ 
      \I_\B(B') = B'\arrow[r,"u_{B'}"]              & \F(A_{B'})  = \F(\G(B')) 
    \end{tikzcd}
    $$
    Hierbei sei $\widetilde f \coloneqq \overline{u_{B'} \circ f} \in [A_B, A_{B'}]_\A$ eindeutig festgelegt über die Eigenschaft der universellen Abbildung $u_B$.
    Wir widerstehen der Versuchung nicht und setzen einfach $\G(f) = \widetilde f$. 
    Tatsächlich haben wir aufgrund der Eindeutigkeitseigenschaft universeller Abbildungen auch gar keine andere Wahl.
    Hiermit ist schonmal folgendes klar: Ist $\G$ ein Funktor, so gibt es keinen weiteren Funktor, der denselben Job erfüllt. 
    Da obiges Diagramm kommutiert, erhalten wir, gesetzt dem Falle, dass $\G$ tatsächlich ein Funktor ist, zudem die im Theorem postulierte Eigenschaft (2).

    Seien also $f,g \in  \Mor_\B$, so erhalten wir das folgende Diagramm:
    $$
        \begin{tikzcd}[column sep=3.5em,row sep=2.5em]
          B \arrow[r,"u_B"] \arrow[d,"f"'] \arrow[dd,"g \circ f"',bend right=60]    & \F(A_B)    \arrow[d,"\F(\widetilde f)"] \arrow[dd, bend left=60, "\F(\widetilde g \circ \widetilde f)"] 
      \\ 
      B'\arrow[r,"u_{B'}"] \arrow[d,"g"'] & \F(A_{B'}) \arrow[d ,"\F(\widetilde g)"] \\
      B''\arrow[r,"u_{B''}"]              & \F(A_{B''}) 
    \end{tikzcd}
    $$
    Dieses Diagramm kommutiert, da $\F$ ein Funktor ist.
    Insbesondere gilt also $\F(\widetilde g) \circ \F(\widetilde f) = \F(\widetilde g \circ \tilde f)$. 
    Andererseits liefert die Definition von $u_B$ auch die Existenz einer Abbildung $\widetilde{g \circ f} = \overline{u_{B''} \circ g \circ f}$ mit 
    $$
    \F(\widetilde{g \circ f}) \circ u_B = g \circ f .
    $$
    Die Eindeutigkeitseigenschaft universeller Abbildungen liefert damit sofort $\widetilde{g \circ f} = \widetilde g \circ \widetilde f$ und folglich auch $\G(g \circ f) = \G(g) \circ \G(f)$.
    Analoge Haarspalterei liefert $\G(1_B) = 1_{A_B}$.  
\end{proof}

So wie ein Ingenieur gerne auf Knöpfe einer unbekannten Machine drückt, so setzt ein Mathematiker gerne mehr oder minder bekannte Morhpismen in neuartige Funktoren ein. Genauer gesagt, interessieren wir uns dafür, welche neuen Morphismen wir erhalten, wenn wir die universellen Abbildungen $(u_B, A_B)$ in den soeben konstruierten Funktor $\G$ einsetzen.

\begin{kor}[\cite{preuss}, 2.1.12]
  Es existiert genau eine natürliche Transformation $v = (v_A) \colon \G \circ F \to \I_A$, sodass das Folgende gilt:
  \begin{enumerate}[(a)]
    \item $\F(v_A) \circ u_{\F(A)} = \1_{\F(A)}$ für alle $A \in |\A|$.
    \item $v_{\G(B)} \circ \G(u_B) = \1_{\G(B)}$ für alle $B \in |\B|$.
  \end{enumerate}
\end{kor}

Wir machen nun den vorangehenden Satz und das zugehörige Korollar zur Definition und untersuchen die dadurch entstehenden Objekte.

\begin{defn}[Linksadjungierter Funktor]
  Sind $\F \colon \A \to \B$ und $\G \colon \B \to \A$ Funktoren und $u = (u_B) \colon \I_\B \to \F \circ \G$ sowie $v = (v_A) \colon \G \circ \F \to \I_\A$ natürliche Transformationen mit den Eigenschaften
  \begin{enumerate}[(1)]
    \item $\F(v_A) \circ u_{\F(A)} = \1_{\F(A)}$ für alle $A \in |\A|$ und
    \item $v_{\G(B)} \circ \G(u_B) = \1_{\G(B)}$ für alle $B \in |\B|$,
  \end{enumerate}
  so nennen wir $\G$ den \emph{zu $\F$ linksadjungierten Funktor} und analog nennen wir $\F$ den \emph{zu $\G$ rechtsadjungierten Funktor}.
  Das Paar $(\G, \F)$ nennen wir ein \emph{Paar adjungierter Funktoren}.
\end{defn}

Haben wir etwa zu viel versprochen?
Dem aufmerksame Leser wird sich wohl kaum ein \emph{Hey, wo sind meine universellen Abbildungen?} verkneifen können.
Keine Angst.
Es stellt sich heraus, dass sich universelle Abbildungen gerne in Rudeln aufhalten, um natürliche Tranformationen zu bilden.

\begin{thm}[\cite{preuss}, 2.1.15]
  Ist $\G \colon \B \to \A$ ein zu $\F \colon \A \to \B$ linksadjungierter Funktor und $u = (u_B) \colon \I_\B \to \F \circ \G$ eine zugehörige natürliche Transformation, dann ist für alle $B \in |B|$ das Paar $(u_B, \G(B))$ eine universelle Abbildung bezüglich $\F$.
\end{thm}

\begin{bem}[Adjungierte Situation]
  Namen sind bekanntlich Schall und Rauch und in der Kategorientheorie nur dann zu gebrauchen, wenn die mit dem Namen assozierten Gebilde auf eine gutartige Weise eindeutig bestimmt sind.
  Dies ist bei adjungierten Funktoren samt Gefolgschaft der Fall:
  \begin{enumerate}[(i)]
    \item Adjungierte Funktoren sind eindeutig bestimmt bis auf natürliche Äquivalenz
    \item Die in der Definition von adjungierten Funktoren auftretenden natürlichen Transformationen sind eindeutig bis auf natürliche Äquivalenz
  \end{enumerate}
  Es macht somit Sinn einem paar adjungierter Funktoren natürliche Transformationen zuzuordnen. Ein so entstehendes Tupel $(\G, \F, u, v)$ wollen wir \emph{adjungierte Situation} taufen.
\end{bem}

\begin{ex}[Stone-\v{C}ech Kompaktifizierung] Der Inklusionsfunktor $\F_e$ aus Beispiel \ref{ex:Example} besitzt also eine Linksadjungierte $\beta \colon \Tych \to \CompHaus$, die jedem Tychonoff-Raum $X$ seine Stone-\v{C}ech-Kompaktifizierung zuordnet. Beim Versuch Tychonoff-Räume zu kompaktifizieren wurde also implizit ein ganzer Funktor mitkonstruiert.
\end{ex}


