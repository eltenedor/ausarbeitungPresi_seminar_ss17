\section*{Einleitung}

So wie innerhalb einer Kategorie einzelne Objekte zueinander über Morphismen in Beziehung stehen, lassen sich auch Kategorien zueinander in Beziehung setzen. Statt Morphismen spricht man in diesem Zusammenhang von Funktoren. Man könnte den Blickwinkel beziehen die Kategorie der Kategorien zu betrachten, in welcher die Morphismen durch Funktoren repräsentiert werden, um zu erkennen dass letztlich wieder dieselbe Frage im Raum steht, nämlich ob zwei Kategorien isomorph sind.

Dies ist nur allzu oft nicht der Fall und macht in gewisser Weise auch den Reiz unterschiedlicher Kategorien aus. In einer ersten Abschwächung der Isomorphie wird man hoffen eine Äquivalenz von Kategorien zu finden. Ist auch das noch zu viel verlangt, so möchte man dennoch versuchen zwei Funktoren, die in entgegengesetzter Richtung zwischen zwei Kategorien operieren, mit den jeweiligen Identitätsfunktoren auf \emph{natürliche} Art in Verbindung zu bringen. Man kommt zum Begriff der Adjunktion, um den es sich in dieser Ausarbeitung drehen wird. Insbesondere werden wir uns mit Inklusionsfunktoren in topologischen Konstrukten beschäftigen und einige der zwischen topologischen Kategorien bestehenden Relationen kategorientheoretisch beschreiben.

Die folgende Ausarbeitung beschränkt sich bis auf ein paar Ausnahmen darauf Resultate aus dem Buch \cite{preuss} zusammenzufassen und hierbei größtenteils auf Beweise zu verzichten. 
Dies sollte keinesfalls auf die Faulheit des Erstellers zurückgeführt werden, sondern eher als Empfehlung verstanden werden, entsprechende Passagen im besagten Buche nachzulesen, da die Beweise dort aus Sicht des Erstellers bereits in einer verständlichen ausführlichen Form vorliegen und eine Aufnahme dieser in die vorliegende Ausarbeitung nicht zu derer Übersichtlichkeit beitragen würde.
In diesem Sinne ist diese Ausarbeitung eher als Wegbeschreibung durch das zweite Kapitel aufzufassen, durchaus aber auch als ein Appetithappen.
