\section{Konvergenzstrukturen und uniforme Konvergenzstrukturen}

In diesem letzten Abschnitt betrachten wir nun unterschiedliche Konvergenzstrukturen und uniforme Strukturen durch die kategorientheoretische Brille, mit dem Ziel diese untereinander in Beziehung zu setzen und die Verbindung von Konvergenzstrukturen und uniformen Strukturen herzustellen.


\subsection{Konvergenzstrukturen}

Zunächst einmal halten wir fest, welche Konvergenzstrukturen für uns interessant sein werden.

\begin{defn}[GKonv und seine Kinder]
  Die Kategorie $\GConv$ der verallgemeinerten Konvergenzräume mit stetigen Abbildungen setzt sich wie folgt zusammen:
  \begin{enumerate}[a)]
    \item Für jede Menge $X$ sei $\FF(X)$ die Menge aller Filter auf $X$. 
      Ein \emph{verallgemeinerter Konvergenzraum} ist ein Paar $(X,q)$, wobei $X$ eine Menge und $q \subset \FF(X) \times X$ eine Relation von Filtern und Punkten (gegen die sie \emph{konvergieren}) ist. 
      Zusätzlich sollen folgende Eigenschaften erfüllt sein:
      \begin{enumerate}[C1)]
        \item $(\dot x, x) \in q$ für alle $x \in X$; \emph{alle Einpunktfilter konvergieren gegen ihren Erzeuger}.
        \item $(\G, x) \in q$, falls $(F,x) \in q$ und $G \supset F$; \emph{Oberfilter konvergenter Filter, erben Grenzwerte}
      \end{enumerate}
    \item Eine Abbildung $f \colon (X,q) \to (X',q')$ zwischen verallgemeinerten Konvergenzräumen heißt \emph{stetig}, falls für alle $(\F,x) \in q$ auch $(f(\F), f(x)) \in q'$ gilt.
  \end{enumerate}
  Ein verallgemeinerter Konvergenzraum heißt
  \begin{enumerate}[a)]
    \item[c)] \emph{Kent Konvergenzraum}, falls folgende Bedingung erfüllt ist:
      \begin{enumerate}
        \item[C3)] $(\F \cap \dot x, x) \in q$, falls $(F, x) \in q$; \emph{Abgeschlossenheit bezüglich endlicher Durchschnitte mit Einpunktfiltern}.
      \end{enumerate}
    \item[d)] \emph{Limesraum}, falls folgende Bedingung erfüllt ist:
      \begin{enumerate}
        \item[C4)] $(\F \cap \G, x) \in q$, falls $(\F,x) \in q$ und $(\G,x) \in q$; \emph{Abgeschlossenheit bezüglich endlicher Durchschnitte}
      \end{enumerate}
    \item[e)] \emph{Pseudotopologischer Raum}, falls folgende Bedingung erfüllt ist:
      \begin{enumerate}
        \item[C5)] $(\F,x) \in q$, falls $(\U, x) \in q$ für alle Ultrafilter $\U \supset \F$.
      \end{enumerate}
    \item[f)] \emph{Prätopologischer Raum}, falls folgende Bedingung erfüllt ist:
      \begin{enumerate}
        \item[C6)] $(\U_q(x), x) \in q$ für alle $x \in X$, wobei $\U_q(x) \coloneqq \bigcap\{ \F \in F(x) \colon (\F, x) \in q\}$
      \end{enumerate}
  \end{enumerate}
  Ein prätopologischer Raum $(X,q)$ heißt 
  \begin{enumerate}[a)]
    \item[g)] \emph{topologischer Raum}, falls die folgende Bedingung erfüllt ist:
      \begin{enumerate}
        \item[C7)] Für alle $U \in \U_q(x)$ existiert ein $V \in \U_q(x)$ sodass $U \in \U_q(y)$ für alle $y \in V$ gilt.
      \end{enumerate}
  \end{enumerate}
\end{defn}

Die eben definierten Klassen definieren volle und unter Isomorphie abgeschlossene Unterkonstrukte von $\GConv$, welche wir im Folgenden mit $\Lim$, $\PsTop$, $\PrTop$ und $\TPrTop$ bezeichnen werden.

\begin{bem}[\cite{preuss}, 2.3.1.2]
Entsprechend der Definitionsreihenfolge existiert auch eine Inklusionskette der definierten Räumlichkeiten:
  $$
  \GConv \supset \KConv \supset \Lim \supset \PsTop \supset \PrTop \supset \TPrTop.
  $$
\end{bem}

\begin{proof}
  Jeder topologische Raum ist per definitionem ein prätopologischer Raum.

  Jeder prätopologische Raum ist ein pseutopologischer Raum: 
  Ist nämlich $(X,q) \in |\PrTop|$, so gilt $(\F,x) \in q$ genau dann, wenn $\F \supset U_q(x)$. Setzen wir nun voraus, dass $(\U,x) \in q$ für alle Ultrafilter $\U \supset \F$ gilt, so folgt aus
  $$
  \U_q(x) \subset \bigcap \{ \U \colon \U \in \FU(\F)\} = \F,
  $$
  wobei $\FU(\F)$ die Menge der Oberultrafiter von $\F$ bezeichne, die Behauptung durch Anwendung von C2).

  Jeder pseutotopologische Raum ist ein Limesraum: 
  Angenommen C4) sei nicht erfüllt für einen Limesraum $(X,q)$, so existieren Filter $\F,\G \in \FF(X)$ mit $(\F,x) \in q$ und $(\G,x) \in q$ aber $(\F \cap \G,x) \not\in q$.
  Folglich besitzt $(\F \cap \G, x)$ nach C5) ein Oberultrafilter $(\U,x) \not\in q$.
  Insbesondere gilt nach C2) $\U \not\supset \F$ und $\U \not\supset \G$, es existieren also $F \in \F$ und $G \in \G$ mit $F,G \not \in \U$.
  Da $\U$ jedoch ein Oberfiter von $\F \cap \G$ ist, enthält er $F \cup G$ und aufgrund der Ultrafiltereigenschaft $F$ oder $G$ im Widerspruch zu $F, G \not\in \U$. 

  Jeder Limesraum ist ein Kent Konvergenzraum: Dies folgt sofort aus C1).

  Dass jeder Kent Konvergenzraum ein verallgemeinerter Konvergenzraum ist, ist wie bei allen anderen Konvergenzstrukturen Teil der Definition.
\end{proof}

\begin{prop}[\cite{preuss}, 2.3.1.3]
  $\KConv$ ist bireflektives und bicoreflektives Unterkonstrukt von $\GConv$.
\end{prop}

\begin{proof}
  Sei $(X,q)$ ein verallgemeinerter Konverenzraum.
  Es lassen sich wie folgt zwei Kent Konvergenzstrukturen $q_r,q_c$ auf $X$ definieren:
  \begin{align*}
    &(\F,x) \in q_r \iff \exists (\G, x) \in q \text{, sodass } \G \cap \dot x \subset \F, \\
    &(\F,x) \in q_c \iff (\F, x) \in q \text{ und } (\F \cap \dot x) \in q.
  \end{align*}
  Es ist klar, dass es sich bei beiden Konvergenzstrukturen, um Kent Konvergenzstrukturen handelt.
  Aus der Konstruktion ergibt sich zudem $q \subset q_r$ und damit $1_X \colon (X,q) \to (X, q_r) \in \Mor_{\GConv}$ sowie $q_s \subset q$ und folglich $1_X \colon (X, q_s) \to (X,q) \in \Mor_{\GConv}$.
  Ist nun $(Y,u) \in |\KConv|$ und $f \colon (X, q) \to (Y,u)$ ein $\GConv$-Morphismus so ist durch $g \coloneqq 1_X^{-1} \circ f$ der gesuchte $\KConv$-Morphismus für die Faktorisierung gegeben.
  Damit ist $1_X$ die gesuchte Bireflektion.
  Analog beweist man den Fall für die Struktur $q_s$, für die $1_X$ zu einer Koreflektion wird.
\end{proof}

\begin{prop}[\cite{preuss}, 2.3.1.5]
  \label{prop:inklusion}
  Jedes der Konstrukte der Inklusionskette
  $$
  \KConv \supset \Lim \supset \PsTop \supset \PrTop \supset \TPrTop.
  $$
  ist ein bireflektives, volles und unter Isomorphie abgeschlossenes Unterkonstrukt der vorangehenden.
\end{prop}

Es mag die Frage aufgekommen sein, weshalb wir die Objekte der Kategorie $\TPrTop$ als topologische Räume bezeichnet haben.
Zwischen besagter Kategorie $\TPrTop$ und $\Top$ besteht eine besondere Art der Isomorphie, die, wie sollte es auch anders sein, sich einen Namen gemacht hat:
\begin{defn}
  \begin{enumerate}[a)]
    \item Seien $\A$ und $\B$ Kategorien.
      Dann nennen wir einen Funktor $\F \colon \A \to \B$ einen \emph{Isomorphismus}, falls ein Funktor $\G \colon \B \to \A$ existiert, sodass $\G \circ \F = \I_\A$ und $\F \circ \G = \I_\B$ gilt.
    \item Seien $\A$ und $\B$ Konstrukte.
      Sind $\Ha \colon \A \to \Set$ und $\K \colon \B \to \Set$ die Vergissfunktoren, so nennen wir einen Funktor $\A \to \B$ \emph{konkret}, falls $\K \circ \F = \Ha$ gilt.
    \item Einen konkreten Funktor $\F \colon \A \to \B$, welcher zudem ein Isomorphismus ist, bezeichnen wir als \emph{konkreten Isomorphismus}.
    \item Wir nennen zwei Kategorien (zwei Konstrukte) $\A$ und $\B$ \emph{isomorph} (\emph{konkret isomorph}), vorausgesetzt es existiert ein Isomorphismus (konkreter Isomorphismus) $\F \colon \A \to \B$.
  \end{enumerate}
\end{defn}

\begin{prop}[\cite{preuss}, 2.3.18]
  $\Top$ und $\TPrTop$ sind konkret Isomorph.
\end{prop}


\subsection{Uniforme Konvergenzstrukturen}

In diesem Abschnitt besprechen wir das zweite für uns interessante topologische Konstrukt mitsamt interessanter Unterkonstrukte.

\begin{defn}[SUConv und Nachfahren]
  Die Kategorie $\SUConv$ der verallgemeinerten Konvergenzräume mit gleichmäßig stetigen Abbildungen setzt sich wie folgt zusammen:
  \begin{enumerate}[a)]
    \item Ein \emph{semiuniformer Konvergenzraum} ist ein Paar $(X, \J_X)$, wobei $X$ eine Menge und $\J_X \subset F(X \times X)$ die Menge der \emph{uniformen Filter} ist mit folgenden Eigenschaften:
      \begin{enumerate}[UC1)]
         \item $(\dot x \times \dot x) \in \J_X$ für alle $x \in X$.
         \item $\G \in \J_X$, falls $\F \in \J_X$ und $\F \subset \G$.
         \item Aus $\F \in \J_X$ folgt $\F^{-1} = \{F^{-1} \colon F \in \F\} \in \J_X$.
      \end{enumerate}
    \item Eine Abbildung $f \colon (X, \J_X) \to (Y, \J_Y)$ zwischen semiuniformen Konvergenzräumen heißt \emph{gleichmäßig stetig}, falls $(f \times f)(\J_X) \subset \J_Y$ gilt.
  \end{enumerate}
  Ein semiuniformer Konvergenzraum heißt 
\begin{enumerate}[a)]
  \item[c)] \emph{semiuniformer Limesraum}, falls die folgende Bedingung erfüllt ist:
    \begin{enumerate}[UC1)]
      \item[UC4)] $F \in \J_X$ und $\G \in \J_X$ implizieren $\F \cap \G \in \J_X$.
    \end{enumerate}
  \item[d)] \emph{uniformer Limesraum}, falls die folgende Bedingung erfüllt ist:
    \begin{enumerate}[UC1)]
      \item[UC5)] $\F \in \J_X$ und $\G \in \J_X$ implizieren $\F \circ \G \in \J_X$.
    \end{enumerate}
  \end{enumerate}
  Ein uniformer Limesraum $(X, \J_X)$ heißt
  \begin{enumerate}[a)]
    \item[e)] \emph{Haupt-uniformer Limesraum} falls eine nichtleere Teilmenge $\F$ von $\Pot(X \times X)$ existiert, welche die endliche Durchschnittseigenschaft besitzt und gegenüber Obermengenbildung abgeschlossen ist und $[F] \coloneqq \{\G \in F(X\times X) \colon \G \supset \F \}= \J_X$ erfüllt.
  \end{enumerate}
\end{defn}

Die eben definierten Klassen definieren volle und unter Isomorphie abgeschlossene Unterkonstrukte von $\SUConv$, welche wir im Folgenden mit $\SULim$, $\ULim$ und $\PrULim$ bezeichnen werden.
Auch für diese Unterkonstrukte existiert eine Inklusionsbeziehung

\begin{prop}[\cite{preuss}, 2.3.2.3]
  Jedes der Konstrukte der Inklusionskette
  $$
  \SUConv \supset \SULim \supset \ULim \supset \PrULim
  $$
  ist ein bireflektives, volles und unter Isomorphie abgeschlossenes Unterkonstrukt der vorangehenden.
\end{prop}


\subsection[Das Bindeglied zwischen beiden Strukturen]{Das Bindeglied zwischen Konvergenzstrukturen und uniformen Konvergenzstrukturen}

Haben wir in den beiden vorangehenden Sektionen Konvergenzstrukturen und uniforme Konvergenzstrukturen getrennt betrachtet, so kümmern wir uns nun darum die Verbindung zwischen beiden Strukturen herzustellen.
Zuvor will jedoch unser Vokabelheft gefüttert werden.

\begin{defn}
  Die Kategorie $\Fil$ der Filterräume und Cauchy stetigen Abbildungen setzt sich wie folgt zusammen:
  \begin{enumerate}[a)]
    \item Ein \emph{Filterraum} ist ein Paar $(X, \gamma)$, wobei $X$ eine Menge und $\gamma$ eine Menge von Filtern ist, sodass die Folgenden Bedingungen erfüllt sind.
      \begin{enumerate}[F1)]
        \item $\dot x \in \gamma$ für alle $x \in X$.
        \item $\G \in \gamma$, falls $\F \in \gamma$ und $\F \subset \G$.
      \end{enumerate}
      Ist $(X,\gamma)$ ein Filterraum, so wollen wir die Elemente von $\gamma$ \emph{Cauchy-Filter} nennen.
    \item Eine Abbildung $f \colon (X, \gamma) \to (X', \gamma')$ zwischen Filterräumen heißt \emph{Cauchy-stetig}, falls $f(\F) \in \gamma'$ gilt, für alle $\F \in \gamma$.
  \end{enumerate}
\end{defn}

\begin{defn}
  Sei $(X, \J_X)$ ein semiuniformer Konvergenzraum.
  Wir nennen einen Filter auf $X$ einen $\J_X$-\emph{Cauchy-Filter}, wenn $\F \times \F \in \J_X$ gilt.
\end{defn}

\begin{defn}
  Wir nennen einen semiuniformen Konvergenzraum $(X, \J_X)$ $\Fil$-bestimmt, falls $\J_X = \J_{\gamma_{J_X}}$ gilt, also $\J_X$ von allen $\J_X$-Cauchy-Filtern erzeugt wird.
\end{defn}

Wer bis hierhin gelesen hat, ahnt bereits, was kommen wird: Unser Bindeglied zwischen Konvergenzstrukturen und uniformen Konvergenzstrukturen werden die Filterräume sein. Das ist schon fast richtig. \emph{Fast} im Sinne von \emph{bis auf konkrete Isomorphie}.


\begin{prop}[\cite{preuss}, 2.3.3.5]
  Ist $\FilDSUConv$ das Konstrukt aller $\Fil$-bestimmten semiuniformen Konvergenzräume (mit gleichmäßig stetigen Abbildungen), dann ist $\Fil$ konkret isomorph zu $\FilDSUConv$.
\end{prop}

Nun geht es ans Eingemachte: Wir stellen die erste Verbindung zu uns bekannten Strukturen her.

\begin{prop}[\cite{preuss}, 2.3.3.6]
  $\FilDSUConv$ ist ein bireflektives und bikoreflektives, volles und unter Isomorphie abgeschlossenes Unterkonstrukt von $\SUConv$.
\end{prop}

Nun steht nur noch offen eine Verbindung zu Konvergenzstrukturen herzustellen.
Dies funktioniert jedoch nicht unmittelbar. 
Um uns im bevorstehenden Terrain weiter bewegen zu können notieren wir in unserem Vokabelheft:

\begin{defn}
  Ein Filterraum $(X, \gamma)$ heißt \emph{vollständig}, falls für alle $\F \in \gamma$ ein $x \in X$ existiert mit $\F \cap \dot x \in \gamma$.
\end{defn}

Damit definieren wir uns nun ein neues Objekt, welches wir in Zusammenhang mit Konvergenzstrukturen bringen werden.

\begin{prop}[\cite{preuss}, 2.3.3.9]
  Das Konstrukt $\CFil$ der vollständigen Filterräume (und Cauchy-stetigen Abbildungen) ist ein volles und unter Isomorphie abgeschlossenes, bicoreflektives Unterkonstrukt von $\Fil$.
\end{prop}

Die Konvergenzstrukturen, die sich mit $\CFil$ in Verbindung bringen lassen, besitzen eine Symmetrieeigenschaft, welche den Zoo der uns bekannten Konvergenzstrukturen mit weiteren bisher unerkannten Arten ausstattet.

\begin{defn}
  Ein verallgemeinerter Konvergenzraum $(X,q)$ heißt \emph{symmetrisch}, falls folgende Bedingung erfüllt ist:
  \begin{enumerate}[S1)]
    \item[S)] $(F,x) \in q$ und $y \in \bigcap_{F \in \F}$ implizieren $(F, y) \in q$.
  \end{enumerate}
\end{defn}

Nun zu dem Kandidaten, welcher Filterräume und Konvergenzräume verbindet.

\begin{prop}[\cite{preuss}, 2.3.3.11]
  \begin{enumerate}[1)]
    \item \begin{enumerate}[a)]  
        \item Sei $(X, \gamma)$ ein Filterraum. Dann wird eine symmetrische Kent Konvergenzstruktur $q_\gamma$ auf $X$ definiert durch
          $$
          (\F, x) \in q_\gamma \iff \F \cap \dot x \in \gamma
          $$
        \item Ist $f \colon (X, \gamma) \to (X', \gamma')$ eine Cauchy-stetige Abbildung zwischen Filterräumen, so ist $f \colon (X, q_\gamma)$ stetig.
      \end{enumerate}
    \item \begin{enumerate}[a)]
        \item Sei $(X, q)$ ein Kent Konvergenzraum. Dann wird eine vollständige $\Fil$-Struktur $\gamma_q$ auf $X$ definiert durch
          $$
          \gamma_q = \{ \F \in \FF(X) \colon \exists x \in X \colon (\F, x ) \in q \}.
          $$
        \item Ist $f \colon (X, q) \to (X', q')$ eine stetige Abbildung zwischen Kent Konvergenzräumen, dann ist $f \colon (X, \gamma_q) \to (X', \gamma_{g'})$ Cauchy-stetig.
      \end{enumerate}
    \item Das Konstrukt $\CFil$ ist konkret isomorph zum Konstrukt $\KConv_\Sy$ der symmetrischen Kent Konvergenzräume (und stetigen Abbildungen).
  \end{enumerate}
\end{prop}

Was hat es nun mit symmetrischen Konvergenzräumen auf sich und wie übertragen sich die Eigenschaften aus Proposition \ref{prop:inklusion} auf ihre symmetrischen Verwandten?

\begin{prop}[\cite{preuss}, 2.3.3.13]
  \begin{enumerate}[1)]
    \item Sei $\A$ ein topologisches Konstrukt der Klasse Limesraum, pseudotopologischer Raum, prätopologischer Raum oder topologischer Raum.
      Dann ist das volle und unter Isomorphie abgeschlossene Unterkonstrukt $\A_\S$ bestehend aus symmetrischen Objekten bireflektiv in $\A$.
    \item Jedes der Konstrukte der Inklusionskette
      $$
      \KConv_\Sy \supset \Lim_\Sy \supset \PsTop_\Sy \supset \PrTop_\Sy \supset \Top_\Sy
      $$
      ist ein bireflektives Unterkonstrukt der Vorgänger.
  \end{enumerate}
\end{prop}

Wir haben unser Ziel erreicht. Das folgende Diagramm liefert eine Zusammenfassung der besprochenen Sachverhalte:
$$
\begin{tikzcd}[column sep=5.5em,row sep=2.5em]
  \SUConv              &        & \GConv \\
  \SULim \arrow[u,"r"] &  \Fil\arrow[lu,"rc"]  & \KConv   \arrow[u,"rc"] \\
  \ULim  \arrow[u,"r"] &        & \KConv_\Sy \arrow[lu,"c"]    \arrow[u,"r"]    \\
  \Unif  \arrow[u,"r"] &        & \Lim_\Sy   \arrow[u,"r"]    \\
                       &        & \PsTop_\Sy \arrow[u,"r"]    \\
                       &        & \PrTop_\Sy \arrow[u,"r"]    \\
                       &        & \Top_\Sy   \arrow[u,"r"]    
\end{tikzcd}
$$
