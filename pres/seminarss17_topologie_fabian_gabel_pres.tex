%Bin kein Freund serifenloser Schriftarten. . . es geht aber sicher noch etwas schöner
\documentclass[serif,9pt]{beamer}
\usepackage[utf8]{inputenc}
\usepackage[T1]{fontenc}
\usepackage{mathtools}
\usepackage[ngerman]{babel}

\usepackage{tikz}
\usepackage{tikz-cd}
\usetikzlibrary{quotes,babel,angles}

\usepackage{lmodern}
\usepackage{anyfontsize}

\setbeamertemplate{theorems}[numbered]

\title[Topologie Seminar]{\huge{Reflektionen \& Coreflektionen}}
\subtitle{Topologie Seminar}
\date[Sommersemester 2017]{Sommersemester 2017}
\author[F. Gabel]{Fabian Gabel}


../ausarbeitung/befehle.tex

\begin{document}

\maketitle

\begin{frame}{Das (sportliche) Programm -- Etappen(ziele)}
  \tableofcontents
\end{frame}

\AtBeginSection[]
{
  \begin{frame}{Inhalt}
    \tableofcontents[currentsection]
  \end{frame}
}

\section{Grundlagen der Kategorientheorie (Teil II)}

\subsection{Funktoren allgemein}

%%%%%%%%%%%%%%%%%%%%%%%%%%%%%%%%%%%%%%%%%%%%%%%%%%%%%%%%%%%%%%%%%%%%%%%%%%%%%%%%
\begin{frame}
  \frametitle{Vokabelheft}
  Objekte verhalten sich zu Morphismen wie
  Kategorien zu \only<1>{???}\only<2,3>{Funktoren.}
  \pause
  \begin{defn*}
  Seien $\C$ und $\De$ Kategorien und $\F_1 \colon |\C| \to |\De|$ and $\F_2 \colon \Mor_\C\to \Mor_\De$. 
  Dann nennen $\F = (\C, \De, \F_1, \F_2)$ einen (\emph{covarianten}) \emph{Funktor} von $\C$ nach $\De$, falls folgende Bedingungen erfüllt sind:
  \begin{enumerate}[F1)]
    \item $f \in [A, B]_\C$ impliziert $\F(f) \in [\F(A), \F(B)]_\De$.
    \item $\F(f \circ g) = \F(f) \circ \F(g)$, falls $f \circ g$ definiert ist.
    \item $\F(1_A) = 1_{\F(A)}$ für alle $A \in |\C|$.
  \end{enumerate}
  Abkürzend: $\F \colon \C \to \De$. (\emph{Homomorphismus von Funktoren})

  \pause

  \emph{Kontravarianter Funktor}, falls modifiziert:
\begin{enumerate}[F1')]
  \item[F2')] $f \in [A, B]_\C$ impliziert $\F(f) \in [\F(B), \F(A)]_\De$.
  \item[F3')] $\F(f \circ g) = \F(g) \circ \F(f)$, falls $f \circ g$ existiert.
\end{enumerate}
  \end{defn*}
\end{frame}

%%%%%%%%%%%%%%%%%%%%%%%%%%%%%%%%%%%%%%%%%%%%%%%%%%%%%%%%%%%%%%%%%%%%%%%%%%%%%%%%
\begin{frame}
  \frametitle{Beispiele}
  \begin{enumerate}[a)]
      \item<+-> Konstanter Funktor:
      $\C,\De$ Kategorien, $X \in |\De|$.

        $\forall A \in |\C|$ und $\forall f \in \Mor_\C$ durch $\F(A) \coloneqq X$ und $\F(f) \coloneqq 1_X$ 
        
        (kovariant und contravariant).
      \item<+-> Vergissfunktor: $\C$ ein (topologisches) Konstrukt: 
        
        $\F \to \Set$ definiert durch $\F((X, \xi)) = X$ und $\F(f) = f$.
      \item<+-> Dualisierender Funktor: 
        
        $\F \colon \C \to \C^*$ definiert durch $\F(X) = X$ und $F(f) = f^*$ (contravariant).
      \item<+-> Dualer Funktor: $\F \colon \C \to \De$ ein Funktor: 
        
        $\F^* \coloneqq \Delta_{\De} \circ F \circ \Delta_{\C^*}$
        
      \item<+-> Identitätsfunktor $\I_\C$: 
        
        $\F \colon \C \to \C$ definiert durch $\F(X) = X$ und $\F(f) = f$.

      \item<+-> Inklusionsfunktor: 
        
        Sei $\C$ eine Kategorie und $\A$ eine \emph{Unterkategorie}, dh 
        \begin{enumerate}[1.]
          \item $|\A| \subset |\C|$,
          \item $[A, B]_\A \subset [A,B]_\C$ für alle $(A, B) \in |\A| \times |\A|$,
          \item Komposition von Mor. in $\A$ wie in $\C$; Identitätsmorphismus derselbe.
        \end{enumerate}
        Gilt sogar $[A, B]_\A = [A, B]_\C$: \emph{volle} Unterkategorie.

        $\F_e \coloneqq \I_\C |_\A$
    \end{enumerate}
\end{frame}

%%%%%%%%%%%%%%%%%%%%%%%%%%%%%%%%%%%%%%%%%%%%%%%%%%%%%%%%%%%%%%%%%%%%%%%%%%%%%%%%
\begin{frame}[fragile]
\frametitle{Definition -- Universelle Abbildung}
  $\A$ und $\B$ Kategorien, $\F \colon \A \to \B$ Funktor und $B \in |\B|$.

  Paar $(u, A)$ mit $A \in |\A|$ und $u \colon B \to \F(A)$ heißt \emph{universelle Abbildung für $B$ bezüglich $\F$}, falls $\forall A' \in |\A|$ und $\forall f \colon B \to \F(A')$ genau ein $\A$-Morphismus $\overline f \colon A \to A'$ existiert so dass das Diagramm
  $$
  \begin{tikzcd}
    B \arrow[rd,"u"] \arrow[rr, "f"] &&\F(A') \\
    &\F(A) \arrow[ru, "\F(\overline{f})"]
  \end{tikzcd}
  $$
  kommutiert.
  \pause

  Entsprechend: Paar $(A,u)$ mit $A \in |\A|$ und $u \colon \F(A) \to B$: 
  \emph{co-universelle Abbildung für $B$ bezüglich $\F$}, falls $(u^*, A)$ eine universelle Abbildung für $B$ bezüglich des Funktors $\F^* \colon \A^* \to \B^*$ ist:
  $$
  \begin{tikzcd}[row sep=2.5em]
    B  &&  \arrow[ll, "f"] \F(A')\arrow[ld, "\F(\overline{f})"]  \\
    & \arrow[lu,"u"]\F(A) & 
  \end{tikzcd}
  $$
  kommutiert.
\end{frame}

%%%%%%%%%%%%%%%%%%%%%%%%%%%%%%%%%%%%%%%%%%%%%%%%%%%%%%%%%%%%%%%%%%%%%%%%%%%%%%%%
\begin{frame}[fragile]
  \frametitle{Das Prinzip bei der Arbeit}
  $$
  \begin{tikzcd}
    B \arrow[rd,"u"] \arrow[rr, "f"] &&\F(A') \\
    &\F(A) \arrow[ru, "\F(\overline{f})"]
  \end{tikzcd}
  $$
  Schonmal gesehen bei der Stone-\v{C}ech-Kompaktifizierung?
  \vspace{1em}
  \pause

  $\F = \F_e \colon \CompHaus \to \Tych$.

  Für alle $X \in \Tych$ ist $(e_x, \beta(X))$ eine universelle Abbildung:

  $Y \in \CompHaus$ und $f \in [X, \F_e(Y)]_{\Tych}$, liefert Satz von Stone-\v{C}ech gerade:
      $$
      \begin{tikzcd}[row sep=2.5em]
        X \arrow[rd,"e_X"] \arrow[rr, "f"] &&\F_e(Y) = Y \\
        &\F_e(\beta(X)) = \beta(X) \arrow[ru, "\F_e(\overline{f})"]
      \end{tikzcd}
      $$
\end{frame}

%%%%%%%%%%%%%%%%%%%%%%%%%%%%%%%%%%%%%%%%%%%%%%%%%%%%%%%%%%%%%%%%%%%%%%%%%%%%%%%%
\begin{frame}[fragile]
  \frametitle{Weitere Beispiele}
  \begin{itemize}
    \item T0-ifizierung
    \item Vergissfunktor
  \end{itemize}
\end{frame}

%%%%%%%%%%%%%%%%%%%%%%%%%%%%%%%%%%%%%%%%%%%%%%%%%%%%%%%%%%%%%%%%%%%%%%%%%%%%%%%%
\begin{frame}[fragile]
  \frametitle{Die \emph{richtigen} Abbildungen zwischen Funktoren}
  Seien $\C$ und $\De$ Kategorien und $\F, \G \colon \C \to \De$ Funktoren.
  \begin{enumerate}[1)]
    \item<+-> Eine Familie $\eta = (\eta_A)_{A \in |\C|}$ mit $\eta_A \in [\F(A), \G(A)]_\De$ für alle $\A \in |\C|$ heißt \emph{natürliche Transformation}, falls für alle $(A,B) \in |\C| \times |\C|$ und alle $f \in [A,B]_\C$ das Diagramm
      $$
      \begin{tikzcd}[row sep=2.5em]
        \F(A)\arrow[r,"\eta_A"]\arrow[d,"\F(f)"] & \G(A)\arrow[d,"\G(f)"] \\
        \F(B)\arrow[r,"\eta_B"] & \G(B)
      \end{tikzcd}
      $$
      kommutiert. Kurz: $\eta \colon \F \to \G$ (Mor. von Funktoren).
    \item<+-> Eine natürliche Transformation $\eta \colon \F \to \G$ heißt \emph{natürliche Äquivalenz}, falls für alle $A \in |\C|$ der Morphismus $\eta_A$ ein Isomorphismus ist.
    \item<+-> $\F$ und $\G$ heißen \emph{natürlich äquivalent},  wenn eine natürliche Äquivalenz $\eta \colon \F \to \G$ existiert. Kurz: $\F \approx \G$.
  \end{enumerate}
\end{frame}

\subsection{Adjungierte Funktoren}

%%%%%%%%%%%%%%%%%%%%%%%%%%%%%%%%%%%%%%%%%%%%%%%%%%%%%%%%%%%%%%%%%%%%%%%%%%%%%%%%
\begin{frame}
  \frametitle{Adjungierter Funktor}

\pause
  \begin{defn*}
  Sind $\F \colon \A \to \B$ und $\G \colon \B \to \A$ Funktoren und $u = (u_B) \colon \I_\B \to \F \circ \G$ sowie $v = (v_A) \colon \G \circ \F \to \I_\A$ natürliche Transformationen mit den Eigenschaften
  \begin{enumerate}[(1)]
    \item $\F(v_A) \circ u_{\F(A)} = \1_{\F(A)}$ für alle $A \in |\A|$ und
    \item $v_{\G(B)} \circ \G(u_B) = \1_{\G(B)}$ für alle $B \in |\B|$,
  \end{enumerate}
  so nennen wir $\G$ den \emph{zu $\F$ linksadjungierten Funktor} und analog nennen wir $\F$ den \emph{zu $\G$ rechtsadjungierten Funktor}.
  Das Paar $(\G, \F)$ nennen wir ein \emph{Paar adjungierter Funktoren}.
\end{defn*}

\pause

  \begin{thm*}
    Ein Funktor $\F \colon \A \to \B$ besitzt einen linksadjungierten Funktor $\G \colon \B \to \A$ genau dann, wenn für alle $B \in |\B|$ eine bezüglich $\F$ universelle Abbildung existiert.
  \end{thm*}

\pause

  \begin{bem}
    Die universelle Abbildung aus dem Satz ist gerade die natürliche Tranformation aus der Definition.
  \end{bem}
\end{frame}

%%%%%%%%%%%%%%%%%%%%%%%%%%%%%%%%%%%%%%%%%%%%%%%%%%%%%%%%%%%%%%%%%%%%%%%%%%%%%%%%
\begin{frame}
  \frametitle{Adjungierte Situation}

  \begin{itemize}
    \item<+-> Adjungierte Funktoren sind bis auf natürliche Äquivalenz eindeutig
    \item<+-> Die natürlichen Transformationen aus der Definition sind als (co-)universelle Abbildungen eindeutig bis auf natürliche Äquivalenz
    \item<+-> Adjungierte Situation: Quadrupel $(\G, \F, u, v)$. 
  \end{itemize}

\pause

  \begin{ex*}
     Wieder Stone-\v{C}ech-Kompaktifizierung:

     $\F = \F_e \colon \CompHaus \to \Tych$.

\pause

     \underline{Für alle} $X \in \Tych$ ist $(e_x, \beta(X))$ eine universelle Abbildung bezüglich $\F_e$
     
     Also: Existiert eine Linksadjungierte $\beta \colon \Tych \to \CompHaus$.
     \vspace{1em}

\pause

     Wir haben (nichts-ahnend) einen Funktor konstruiert!
  \end{ex*}

\end{frame}


\section{Reflektive und coreflektive Unterkategorien}

\subsection{Allgemein}

%%%%%%%%%%%%%%%%%%%%%%%%%%%%%%%%%%%%%%%%%%%%%%%%%%%%%%%%%%%%%%%%%%%%%%%%%%%%%%%%
\begin{frame}
  \frametitle{Definition -- Reflektive Unterkategorie}
  $A$ Unterkategorie einer Kategorie $\C$.
  
  $\F_e \colon \A \to \C$ der Inklusionsfunktor.

  Dann nennen wir $\A$ \emph{reflektiv} in $\C$ genau dann, wenn eine der folgenden äquivalenten Bedingungen erfüllt ist:
  \begin{enumerate}[(1)]
    \item $\F_e$ besitzt einen linksadjungierten Funktor $\R$.
    \item Für alle $X \in |C|$ ex. eine universelle Abbildung $(r_X, X_\A)$ bezüglich $\F_e$.
  \end{enumerate}

  \pause
  \vspace{1em}
  Funktor $\R$ nennen wir \emph{Reflektor}
  
  Morphismen $r_X \colon X \to X_\A$ nennen wir Reflektionen von $X$ bezüglich $\A$.

  \pause
  \vspace{1em}
  Durch Dualisierung erhalten wir einen weiteren Begriff:

  Wir nennen $\A$ \emph{coreflektiv} in $\C$, genau dann, wenn $\A^*$ reflektiv ist in $\C^*$. 

  \pause
  \vspace{1em}

  Wir nennen $\A$ \emph{epireflektiv/ extremal epireflektiv/ bireflektiv in} $\C$, falls 
  \pause
  \begin{itemize}
    \item  $\A$ reflektiv in $\C$
    \item  $r_X \colon X \to X_\A$ ist ein Epimorphismus/ extremaler Epimorphismus / Bimorphismus ist.
  \end{itemize}
   
  Die Morphismen $r_X$ nennen wir \emph{Epireflektionen/ extremale Epireflektionen/ Bireflektionen}.
\end{frame}

%%%%%%%%%%%%%%%%%%%%%%%%%%%%%%%%%%%%%%%%%%%%%%%%%%%%%%%%%%%%%%%%%%%%%%%%%%%%%%%%
\begin{frame}
\frametitle{Wie sehen Bireflektionen oder Bicoreflektionen denn aus?}
\end{frame}

\subsection{In topologischen Konstrukten}

%%%%%%%%%%%%%%%%%%%%%%%%%%%%%%%%%%%%%%%%%%%%%%%%%%%%%%%%%%%%%%%%%%%%%%%%%%%%%%%%
\begin{frame}
  \frametitle{Was ist so toll an bireflektiven oder bicoreflektiven Unterkategorien?} 

\pause
  Bi(co)reflektive Unterkategorien sind gutartig im folgenden Sinne:
\pause

\begin{thm*}
  Ist $\A$ ein 
\begin{itemize}
\item volles,
\item unter Isomorphie abgeschlossenes
\item Unterkonstrukt eines topologischen Konstrukts $\C$.
\end{itemize}

\vspace{1em}
\pause
  und ist $\A$ bireflektiv (bicoreflektiv) in $\C$, dann:

\begin{itemize}
  \item<+-> $\A$ ist topologisch.
\item<+-> initialen (finalen) Strukturen in $\A$ stimmen mit denen in $\C$ überein.
\item<+-> finale (initiale) Strukturen in $\A$ entstehen aus den \\
finalen (initialen) Strukturen in $\C$, indem man den \\Bireflektor (Bicoreflektor) anwendet.
\end{itemize}
\end{thm*}

\pause
\vspace{1em}
\huge{Oha!}

\end{frame}

%%%%%%%%%%%%%%%%%%%%%%%%%%%%%%%%%%%%%%%%%%%%%%%%%%%%%%%%%%%%%%%%%%%%%%%%%%%%%%%%
\begin{frame}
\frametitle{Beweis}
Sei $\A$ bicoreflektiv in $\C$...
\pause
\begin{enumerate}[(1)]
\item<+-> $\A$ ist wieder topologisch:
\begin{itemize}
  \item<+-> Existenz und Eindeutigkeit initialer Strukturen:

  Daten: $X$ Menge,$((X_i, \xi_i))_{i \in I}$ Familie von $\A$-Objekten. 

  $\xi$ die initiale $\C$-Struktur auf $X$. 

  Zurückholen der $\C$-Struktur durch Bikoreflektor:
$$
\1_X \colon (X, \xi_\A) \to (X, \xi) 
$$
Zeige nun: $\xi_\A$ ist eindeutige Initialstruktur auf $\A$ ist...

\item<+-> Für alle $X$ ist $\{(Y, \eta) \in |\A| \colon Y = X \} \subset\{(Z, \zeta) \in |\C| \colon Z = X \}$ Menge.

\item<+-> $X$ einelementig: Nur diskrete Struktur und diese ist eindeutig.
\end{itemize}

\item<+-> Bildung finaler Strukturen:
\begin{itemize}
\item<+-> $X$ Menge, $((X_i, \xi_i))_{i \in I}$ Familie von $\A$-Objekten. 
\item<+-> $\xi_\A$ die finale $\A$-Struktur und $\xi_\C$ die finale $\C$-Struktur bzgl. d. Dat.
\item<+-> $\xi_\A = \xi_\C$... \pause !
\end{itemize}

\end{enumerate}

\pause
\vspace{2em}
Sei $\A$ bireflektiv in $\C$... Analog.

\end{frame}

\section{Konvergenzstrukturen und uniforme Konvergenzstrukturen}

\subsection{Konvergenzstrukturen}
\subsection{Uniforme Konvergenzstrukturen}
\subsection{Bindeglied zwischen beiden Strukturen}

\end{document}
