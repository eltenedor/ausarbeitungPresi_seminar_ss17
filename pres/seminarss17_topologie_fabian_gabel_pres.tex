%Bin kein Freund serifenloser Schriftarten. . . es geht aber sicher noch etwas schöner
\documentclass[serif,9pt]{beamer}
\usepackage[utf8]{inputenc}
\usepackage[T1]{fontenc}
\usepackage{mathtools}
\usepackage[ngerman]{babel}

\usepackage{tikz}
\usepackage{tikz-cd}
\usetikzlibrary{quotes,babel,angles}

\usepackage{lmodern}
\usepackage{anyfontsize}

\setbeamertemplate{theorems}[numbered]

\title[Topologie Seminar]{\huge{Reflektionen \& Coreflektionen}}
\subtitle{Topologie Seminar}
\date[Sommersemester 2017]{Sommersemester 2017}
\author[F. Gabel]{Fabian Gabel}


\input{befehle}

\begin{document}

\maketitle

\begin{frame}{Das (sportliche) Programm -- Etappen(ziele)}
  \tableofcontents
\end{frame}

\AtBeginSection[]
{
  \begin{frame}{Inhalt}
    \tableofcontents[currentsection]
  \end{frame}
}

\section{Kategorientheorie Grundlagen (Teil II)}

\subsection{Funktoren allgemein}

%%%%%%%%%%%%%%%%%%%%%%%%%%%%%%%%%%%%%%%%%%%%%%%%%%%%%%%%%%%%%%%%%%%%%%%%%%%%%%%%
\begin{frame}
  \frametitle{Vokabelheft}
  Objekte verhalten sich zu Morphismen wie
  Kategorien zu ???
  \pause
  \begin{defn*}
  Seien $\C$ und $\De$ Kategorien und $\F_1 \colon |\C| \to |\De|$ and $\F_2 \colon \Mor_\C\to \Mor_\De$. 
  Dann nennen $\F = (\C, \De, \F_1, \F_2)$ einen (\emph{covarianten}) \emph{Funktor} von $\C$ nach $\De$, falls folgende Bedingungen erfüllt sind:
  \begin{enumerate}[F1)]
    \item $f \in [A, B]_\C$ impliziert $\F(f) \in [\F(A), \F(B)]_\De$.
    \item $\F(f \circ g) = \F(f) \circ \F(g)$, falls $f \circ g$ definiert ist.
    \item $\F(1_A) = 1_{\F(A)}$ für alle $A \in |\C|$.
  \end{enumerate}
  Abkürzend: $\F \colon \C \to \De$. (\emph{Homomorphismus von Funktoren})

  \pause

  \emph{Kontravarianter Funktor}, falls modifiziert:
\begin{enumerate}[F1')]
  \item[F2')] $f \in [A, B]_\C$ impliziert $\F(f) \in [\F(B), \F(A)]_\De$.
  \item[F3')] $\F(f \circ g) = \F(g) \circ \F(f)$, falls $f \circ g$ existiert.
\end{enumerate}
  \end{defn*}
\end{frame}

%%%%%%%%%%%%%%%%%%%%%%%%%%%%%%%%%%%%%%%%%%%%%%%%%%%%%%%%%%%%%%%%%%%%%%%%%%%%%%%%
\begin{frame}
  \frametitle{Beispiele}
  \begin{enumerate}[a)]
      \item<+-> Konstanter Funktor:
      $\C,\De$ Kategorien, $X \in |\De|$.

        $\forall A \in |\C|$ und $\forall f \in \Mor_\C$ durch $\F(A) \coloneqq X$ und $\F(f) \coloneqq 1_X$ 
        
        (kovariant und contravariant).
      \item<+-> Vergissfunktor: $\C$ ein (topologisches) Konstrukt: 
        
        $\F \to \Set$ definiert durch $\F((X, \xi)) = X$ und $\F(f) = f$.
      \item<+-> Dualisierender Funktor: 
        
        $\F \colon \C \to \C^*$ definiert durch $\F(X) = X$ und $F(f) = f^*$ (contravariant).
      \item<+-> Dualer Funktor: $\F \colon \C \to \De$ ein Funktor: 
        
        $\F^* \coloneqq \Delta_{\De} \circ F \circ \Delta_{\C^*}$
        
      \item<+-> Identitätsfunktor $\I_\C$: 
        
        $\F \colon \C \to \C$ definiert durch $\F(X) = X$ und $\F(f) = f$.

      \item<+-> Inklusionsfunktor: 
        
        Sei $\C$ eine Kategorie und $\A$ eine \emph{Unterkategorie}, dh 
        \begin{enumerate}[1.]
          \item $|\A| \subset |\C|$,
          \item $[A, B]_\A \subset [A,B]_\C$ für alle $(A, B) \in |\A| \times |\A|$,
          \item Komposition von Mor. in $\A$ wie in $\C$; Identitätsmorphismus derselbe.
        \end{enumerate}
        Gilt sogar $[A, B]_\A = [A, B]_\C$: \emph{volle} Unterkategorie.

        $\F_e \coloneqq \I_\C |_\A$
    \end{enumerate}
\end{frame}

%%%%%%%%%%%%%%%%%%%%%%%%%%%%%%%%%%%%%%%%%%%%%%%%%%%%%%%%%%%%%%%%%%%%%%%%%%%%%%%%
\begin{frame}[fragile]
\frametitle{Definition -- Universelle Abbildung}
  $\A$ und $\B$ Kategorien, $\F \colon \A \to \B$ Funktor und $B \in |\B|$.

  Paar $(u, A)$ mit $A \in |\A|$ und $u \colon B \to \F(A)$ heißt \emph{universelle Abbildung für $B$ bezüglich $\F$}, falls $\forall A' \in |\A|$ und $\forall f \colon B \to \F(A')$ genau ein $\A$-Morphismus $\overline f \colon A \to A'$ existiert so dass das Diagramm
  $$
  \begin{tikzcd}
    B \arrow[rd,"u"] \arrow[rr, "f"] &&\F(A') \\
    &\F(A) \arrow[ru, "\F(\overline{f})"]
  \end{tikzcd}
  $$
  kommutiert.
  \pause

  Entsprechend: Paar $(A,u)$ mit $A \in |\A|$ und $u \colon \F(A) \to B$: 
  \emph{co-universelle Abbildung für $B$ bezüglich $\F$}, falls $(u^*, A)$ eine universelle Abbildung für $B$ bezüglich des Funktors $\F^* \colon \A^* \to \B^*$ ist:
  $$
  \begin{tikzcd}[row sep=2.5em]
    B  &&  \arrow[ll, "f"] \F(A')\arrow[ld, "\F(\overline{f})"]  \\
    & \arrow[lu,"u"]\F(A) & 
  \end{tikzcd}
  $$
  kommutiert.
\end{frame}

%%%%%%%%%%%%%%%%%%%%%%%%%%%%%%%%%%%%%%%%%%%%%%%%%%%%%%%%%%%%%%%%%%%%%%%%%%%%%%%%
\begin{frame}[fragile]
  \frametitle{Das Prinzip bei der Arbeit}
  $$
  \begin{tikzcd}
    B \arrow[rd,"u"] \arrow[rr, "f"] &&\F(A') \\
    &\F(A) \arrow[ru, "\F(\overline{f})"]
  \end{tikzcd}
  $$
  Schonmal gesehen bei der Stone-\v{C}ech-Kompaktifizierung?
  \vspace{1em}

  $\F = \F_e \colon \CompHaus \to \Tych$.

  Für alle $X \in \Tych$ ist $(e_x, \beta(X))$ eine universelle Abbildung:

  $Y \in \CompHaus$ und $f \in [X, \F_e(Y)]_{\Tych}$, liefert Satz von Stone-\v{C}ech gerade:
      $$
      \begin{tikzcd}[row sep=2.5em]
        X \arrow[rd,"e_X"] \arrow[rr, "f"] &&\F_e(Y) = Y \\
        &\F_e(\beta(X)) = \beta(X) \arrow[ru, "\F_e(\overline{f})"]
      \end{tikzcd}
      $$
\end{frame}

%%%%%%%%%%%%%%%%%%%%%%%%%%%%%%%%%%%%%%%%%%%%%%%%%%%%%%%%%%%%%%%%%%%%%%%%%%%%%%%%
\begin{frame}[fragile]
  \frametitle{Weitere Beispiele}
  \begin{itemize}
    \item T0-ifizierung
    \item Vergissfunktor
  \end{itemize}
\end{frame}

\subsection{Adjungierte Funktoren}

\section{Reflektive und coreflektive Unterkategorien}

%%%%%%%%%%%%%%%%%%%%%%%%%%%%%%%%%%%%%%%%%%%%%%%%%%%%%%%%%%%%%%%%%%%%%%%%%%%%%%%%


\subsection{Allgemein}
\subsection{In topologischen Konstrukten}

\section{Konvergenzstrukturen und uniforme Konvergenzstrukturen}

\subsection{Konvergenzstrukturen}
\subsection{Uniforme Konvergenzstrukturen}
\subsection{Bindeglied zwischen beiden Strukturen}

\end{document}
