\section{Reflektive und coreflektive Unterkategorien}

\subsection{Allgemein}

%%%%%%%%%%%%%%%%%%%%%%%%%%%%%%%%%%%%%%%%%%%%%%%%%%%%%%%%%%%%%%%%%%%%%%%%%%%%%%%%
\begin{frame}
  \frametitle{Definition -- Reflektive Unterkategorie}
  $A$ Unterkategorie einer Kategorie $\C$.
  
  $\F_e \colon \A \to \C$ der Inklusionsfunktor.

  Dann nennen wir $\A$ \emph{reflektiv} in $\C$ genau dann, wenn eine der folgenden äquivalenten Bedingungen erfüllt ist:
  \begin{enumerate}[(1)]
    \item $\F_e$ besitzt einen linksadjungierten Funktor $\R$.
    \item Für alle $X \in |C|$ ex. eine universelle Abbildung $(r_X, X_\A)$ bezüglich $\F_e$.
  \end{enumerate}

  \pause
  \vspace{1em}
  Funktor $\R$ nennen wir \emph{Reflektor}
  
  Morphismen $r_X \colon X \to X_\A$ nennen wir Reflektionen von $X$ bezüglich $\A$.

  \pause
  \vspace{1em}
  Durch Dualisierung erhalten wir einen weiteren Begriff:

  Wir nennen $\A$ \emph{coreflektiv} in $\C$, genau dann, wenn $\A^*$ reflektiv ist in $\C^*$. 

  \pause
  \vspace{1em}

  Wir nennen $\A$ \emph{epireflektiv/ extremal epireflektiv/ \textbf{bireflektiv} in} $\C$, falls 
  \pause
  \begin{itemize}
    \item  $\A$ reflektiv in $\C$
    \item  $r_X \colon X \to X_\A$ ist ein Epimorphismus/ extremaler Epimorphismus / Bimorphismus ist.
  \end{itemize}
   
  Die Morphismen $r_X$ nennen wir \\\emph{Epireflektionen/ extremale Epireflektionen/ \textbf{Bireflektionen}}.
\end{frame}

%%%%%%%%%%%%%%%%%%%%%%%%%%%%%%%%%%%%%%%%%%%%%%%%%%%%%%%%%%%%%%%%%%%%%%%%%%%%%%%%
\begin{frame}
\frametitle{Wie sehen Bireflektionen oder Bicoreflektionen denn aus?}
\end{frame}

\subsection{In topologischen Konstrukten}

%%%%%%%%%%%%%%%%%%%%%%%%%%%%%%%%%%%%%%%%%%%%%%%%%%%%%%%%%%%%%%%%%%%%%%%%%%%%%%%%
\begin{frame}
  \frametitle{Was ist so toll an bireflektiven oder bicoreflektiven Unterkategorien?} 

\pause
  Bi(co)reflektive Unterkategorien sind gutartig im folgenden Sinne:
\pause

\begin{thm*}
  Ist $\A$ ein 
\begin{itemize}
\item volles,
\item unter Isomorphie abgeschlossenes
\item Unterkonstrukt eines topologischen Konstrukts $\C$.
\end{itemize}

\vspace{1em}
\pause
  und ist $\A$ bireflektiv (bicoreflektiv) in $\C$, dann:

\begin{itemize}
  \item<+-> $\A$ ist topologisch.
\item<+-> initialen (finalen) Strukturen in $\A$ stimmen mit denen in $\C$ überein.
\item<+-> finale (initiale) Strukturen in $\A$ entstehen aus den \\
finalen (initialen) Strukturen in $\C$, indem man den \\Bireflektor (Bicoreflektor) anwendet.
\end{itemize}
\end{thm*}

\pause
\vspace{1em}
\huge{Oha!}

\end{frame}

%%%%%%%%%%%%%%%%%%%%%%%%%%%%%%%%%%%%%%%%%%%%%%%%%%%%%%%%%%%%%%%%%%%%%%%%%%%%%%%%
\begin{frame}
\frametitle{Beweis}
Sei $\A$ bicoreflektiv in $\C$...
\pause
\begin{enumerate}[(1)]
\item<+-> $\A$ ist wieder topologisch:
\begin{itemize}
  \item<+-> Existenz und Eindeutigkeit initialer Strukturen:

  Daten: $X$ Menge,$((X_i, \xi_i))_{i \in I}$ Familie von $\A$-Objekten. 

  $\xi$ die initiale $\C$-Struktur auf $X$. 

  Zurückholen der $\C$-Struktur durch Bikoreflektor:
$$
\1_X \colon (X, \xi_\A) \to (X, \xi) 
$$
Zeige nun: $\xi_\A$ ist eindeutige Initialstruktur auf $\A$ ist...

\item<+-> Für alle $X$ ist $\{(Y, \eta) \in |\A| \colon Y = X \} \subset\{(Z, \zeta) \in |\C| \colon Z = X \}$ Menge.

\item<+-> $X$ einelementig: Nur diskrete Struktur und diese ist eindeutig.
\end{itemize}

\item<+-> Bildung finaler Strukturen:
\begin{itemize}
\item<+-> $X$ Menge, $((X_i, \xi_i))_{i \in I}$ Familie von $\A$-Objekten. 
\item<+-> $\xi_\A$ die finale $\A$-Struktur und $\xi_\C$ die finale $\C$-Struktur bzgl. d. Dat.
\item<+-> $\xi_\A = \xi_\C$... \pause !
\end{itemize}

\end{enumerate}

\pause
\vspace{2em}
Sei $\A$ bireflektiv in $\C$... Analog.

\end{frame}
